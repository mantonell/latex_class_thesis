\newcommand{\eqdef}{\stackrel{\textrm{def}}{=}}
\newcommand{\1}{\mathbbm{1}}
\newcommand{\pr}{\mathbb{P}}
\newcommand{\R}{\mathbb{R}}
\newcommand{\wlo}{\underline{w}}
\newcommand{\wup}{\overline{w}}
\newcommand{\npart}[1]{ \left( #1 \right)^- }
\newcommand{\nmaxpart}[1]
	{ \max \left\{ 0, - \left( #1 \right) \right\} }
\newcommand{\ppart}[1]{ \left( #1 \right)^+ }
\newcommand{\pmaxpart}[1]
	{ \max \left\{ 0, #1 \right\} }
	\newcommand{\dd}[1]{\textrm{d}#1}
\newcommand{\T}{\mathcal{T}}
\newcommand{\csetAll}{\mathcal{U}_{\textrm{re}} \times
					\mathcal{U}_{\textrm{inv}}}

\newcommand{\csetRe}{\mathcal{U}_{\textrm{re}}}

\newcommand{\csetInv}{\mathcal{U}_{\textrm{inv}}}	

\newcommand{\Exp}[1]{\text{Exp}(#1)}				
					
% Solita funzione indicatrice
\newcommand{\indS}{\1_{t + S \leq T}}	

% Speranza con apici e pedici					
\DeclareDocumentCommand \E 
	{ O{} O{} m } {
	\mathbb{E}^{#1}_{#2} 
	\left[
	#3
	\right]
}

% Operatore T
\DeclareDocumentCommand \T
	{ m o } {
	\IfNoValueTF{#2}
		{ \mathcal{T}^{#1} }
		{ \mathcal{T}^{#1} \left[ #2 \right] }
}

% Valore assoluto e norme
\DeclareDocumentCommand \abs 
	{ m } {
	\left| #1 \right|
}
\DeclareDocumentCommand \norm
	{ m O{} } {
	\left|\left| #1 \right| \right|_{#2}	
}

% Operatore T
\DeclareDocumentCommand \A
	{ m m o } {
	\IfNoValueTF{#3}
		{ \mathbf{A}^{#1}_{#2} }
		{ \mathbf{A}^{#1}_{#2} \left[ #3 \right] }
}

% Funzione Pi
\DeclareDocumentCommand \fp
	{ m m o } {
	\IfNoValueTF{#3}
		{ \Pi^{#1}_{#2} }
		{ \Pi^{#1}_{#2} \left[ #3 \right] }
}

% funzioni a^{m,k}_1
\DeclareDocumentCommand \aa
	{ m m }{ a^{#1}_{#2} }
	
% bgrid
%\newcommand{\bg}{b^{\textrm{grid}}}

% value @ risk
\newcommand{\VaR}{\textrm{VaR}}
					